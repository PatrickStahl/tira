\chapter{Virtual Machines}

\section{Prepare Guest OS}
This section describes which configurations have been done for preparing the os of the virtual machines (guest os).
\subsection*{Ubuntu}
After the setup of an new VM with a plain Ubuntu the following steps have been done.
\begin{enumerate}
\item Install guest additions:
\begin{itemize}
\item Download: \cmd{download.virtualbox.org/virtualbox/4.2.10/VBoxGuestAdditions\_4.2.10.iso}
\item Install: \cmd{http://www.virtualbox.org/manual/ch04.html}
\end{itemize} 
\item Install openssh-server:\\
\cmd{sudo apt-get install openssh-server}
\item Install virtualbox packages to ensure working guest additions after kernel update:\\ 
\cmd{sudo apt-get install virtualbox-ose-guest-utils virtualbox-ose-guest-x11 virtualbox-ose-guest-dkms}
\item Set Shared Folder Mount Prefix:\\
\cmd{VBoxManage guestproperty set <vm-name> /VirtualBox/GuestAdd/SharedFolders/MountPrefix "" }
\end{enumerate}

\subsection*{Windows}
After the setup of an new VM with a plain Windows the following steps have been done.
\begin{enumerate}
\item Activate admin account:\\
\cmd{net user administrator /active:yes}
\item Install guest additions:
\begin{itemize}
\item Download: \cmd{download.virtualbox.org/virtualbox/4.2.10/VBoxGuestAdditions\_4.2.10.iso}
\item Install: \cmd{http://www.virtualbox.org/manual/ch04.html}
\end{itemize} 
\item Install SSH server Cygwin 
\begin{itemize}
	\item Download and start: \cmd{www.cygwin.com/setup.exe}
	\begin{itemize}
		\item Download Site e.g.: http:linux.rz.ruhr-uni-bochum.de, or uni-erlangen
		\item Select packages: Net $\rightarrow$ openssh
	\end{itemize}
	\item Start cygwin as administrator (right click \"Run as administrator\")
	\item Run \cmd{ssh-host-config}
	\begin{itemize}
    	\item Query: Should privilege separation be used? (yes/no): yes
    	\item Query: New local account 'sshd'? (yes/no): yes
    	\item Query: Do you want to install sshd as a service?
    	\item Query: Say "no" if it is already installed as a service (yes/no): yes
    	\item Query: Enter the value of CYGWIN for the deamon: [] xterm
    	\item Query: Do you want to use a different name? (yes/no) no
    	\item Enter Administrator password as privileged server password.
	\end{itemize}
	\item Set home directory of user in \cmd{C:/cygwin/etc/passwd /cygdrive/c/Users/<username>}
	\item Run \cmd{net start sshd}
	\item To reconfigure sshd, type:\\
		\cmd{cygrunsrv --stop sshd}\\
		\cmd{cygrunsrv --remove sshd}
\end{itemize}

\item Install software (jdk, python, etc...)
% OLD STUFF
%# SSH on windows
%http://www.worldgoneweb.com/2011/installing-openssh-on-windows-7/
%cmd: cd C:\Program Files (x86)\OpenSSH\etc\
% mkpasswd -l -u administrator -p "\cygdrive\c\users" >> etc\passwd
%
%# enable users ssh connection
%mkpasswd -l -u <user> >> etc/passwd
%
%# start ssh server
%net start opensshd
\end{enumerate}

\section{Virtual Machines}\label{ova}
Each new virtual machine (VM) will be instantiated by using a VirtualBox image file (*.ova).
Find the ova-files with operating systems Ubuntu Desktop~12.04~(64bit) and Microsoft Windows~7~(64bit) on \cmd{webis16}:\\
\cmd{/media/storage2/data-in-progress/workshop-PAN-13/pan13-virtual-machines-ova-files}:
\begin{itemize}
\item \texttt{tira-ubuntu-12-04-desktop-64bit.ova}
\begin{itemize}
\item OS: Linux
\item Version: Ubuntu 12.04 (64 bit)
\item Base Memory Size: 4096 MB
\item Storage: 16 GB (Dynamically allocated)
\item Media Source: ubuntu-12.04.1-desktop-amd64.iso
\item Ova-Image: 2.8 GB
\end{itemize}
\item \texttt{tira-windows-7-64bit.ova}
\begin{itemize}
\item OS: Windows
\item Version: Windows 7 (64 bit)
\item Base Memory Size: 4096 MB
\item Storage Details: 25 GB (Dynamically allocated)
\item Media Source: windows 7 (with kms-activation see~\ref{kms})
\item Ova-Image: 5.4 GB
\end{itemize}
\end{itemize}
This files will be used for setting up a VM for a participant (see \ref{setup}).


\section{How to list existing VMs on a host-pc?}\label{list}
First, connect to the host-pc via ssh. In order to list all existing VMs on host execute the following command:\\
\cmd{VBoxManage list vms}
\medskip\noindent

In order to list all running VMs:\\
\cmd{VBoxManage list runningvms}\\
The output of both commands will look like this:
\lstinputlisting[style=BashOutputStyle]{example-list-runningvms.txt}
Where the first string is name of the VM and the second string in $\{...\}$ is the UUID (unique identifier) of the VM.




\section{How to setup a VM for a participant?}\label{setup}
Everything you need to know for setting up a VM for a participant is the name of the \textit{host-pc}, the preferred \textit{operating system} and the \textit{user name}. In order to setup a VM follow these 4~steps:
\begin{enumerate}
\item Connect to a host pc via ssh as webis user:\\
\cmd{ssh webis@<host-pc>}
\item Change directory:\\
\cmd{cd tira-8/virtualbox/vm-conf}
\item Execute configuration script:\\
\cmd{python configure-vm.py <ova-file> <user-name>}\\
Where \texttt{<ova>} is the ova-file of the preferred os (\ref{ova}) and \texttt{<user-name>} is the user name of the participant. The configuration script will output several lines.
After some lines of:\\
\cmd{ssh: connect to host 10.<host-id>.<vm-id>.100 port 22: No route to host}\\
the final output will look like this:\\
\lstinputlisting[style=BashOutputStyle]{example-vm-conf-output.txt}
\medskip\noindent
The VM will be started automatically by the configuration script. Since not the whole configuration process can be automated, you need to follow some steps to finalize the process and test the VM. Please follow these steps depending on the VMs os:

\subsection*{Ubuntu}

\item Ensure that the VM is running by checking the list of running VMs (see \ref{list}):
\cmd{VBoxManage list runningvms}
\item Connect via remote desktop connection (see \ref{access}) and login as participant which creates the participant's home directory:\\
\cmd{rdesktop <host-pc>:<port-rdp> -u <user-name> -p <user-password>}\\
This command is already prepared by the configuration script's output.\\
Cancel the connection by closing the remote connection window.
\item Test the ssh connection with the participant's user account (see \ref{access}):\\
\cmd{ssh <user-name>@<host-pc> -p <port-ssh>}\\
In order to not enter the cryptic password interactively, use the command prepared by the configuration script's output.\\
Cancel the connection with the \texttt{exit} command
\end{enumerate}

\begin{enumerate}
\setcounter{enumi}{3}
\subsection*{Windows}
\item Ensure that the VM is running by checking the list of running VMs (see \ref{list}):
\cmd{VBoxManage list runningvms}
\item Connect via remote desktop connection (see \ref{access}):\\
\cmd{rdesktop <host-pc>:<port-rdp> -u <user-name> -p <user-password>}\\
This command is already prepared by the configuration script's output.\\
Now, you see the Windows login screen with only the administrator account listed.
\item Login as administrator. The administrator password is provided in the configuration script's output.
\item Open the Windows terminal as administraor (right-click: ``Run as Administrator'') and execute the following commands in order to activate the Windows license:\\
\cmd{cscript.exe c:\textbackslash windows\textbackslash system32\textbackslash slmgr.vbs -skms brocken.rz.tu-ilmenau.de}\\
\cmd{cscript.exe c:\textbackslash windows\textbackslash system32\textbackslash slmgr.vbs -ato}
\item Log off from the administrator account. Now, you can see the participant's user account on the Windows login screen as well.
\item Login as participant which creates the participant's home directory. The participant's password is provided in the configuration script's output. 
\item Log off from the participant's account and cancel the remote desktop connection by closing the remote desktop connection window.
\item Test the ssh connection with the participant's user account (see \ref{access}):\\
\cmd{ssh <user-name>@<host-pc> -p <port-ssh>}\\
In order to not enter the cryptic password interactively, use the command prepared by the configuration script's output.\\
Cancel the connection with the \texttt{exit} command
\end{enumerate}

After successfully setting up the participant's VM please save the access information by copying the configuration script's summary to the file \notes.


\section{How to access the VM of a participant?}\label{access}
First, you need the access information of the participant's VM which are listed in the file \notes.
Now, you can access the VM of a participant via ssh or remote desktop. 
\medskip\noindent

Access a VM via ssh connection with the following command:\\
\cmd{ssh <user-name>@<host-pc> -p <port-ssh>}\\
Cancel the connection with the following command:\\
\texttt{exit}
\medskip\noindent

Access a VM via remote desktop with the following command:\\
\cmd{rdesktop <host-pc>:<port-rdp> -u <user-name> -p <user-password>}\\\\
Cancel the connection by closing the remote connection window.


\section{How to start an existing VM?}\label{start}
Since our host-pc are headless server the VM type headless is required. Start an existing virtual machine by executing the following command on the host pc:\\
\cmd{VBoxManage startvm <vm-name> --type headless}


\section{How to stop a running VM?}\label{stop}
Stop a VM by executing the following command on the host pc:\\
\cmd{VBoxManage controlvm <vm-name> poweroff}
\medskip\noindent

Alternatively, you can connect via remote desktop and shutdown the VM over its common shutdown menu.

\section{How to delete a VM?}
Before you can delete a VM you have to stop it first(see \ref{stop}). Delete a VM from host pc by executing the following command:\\
\cmd{VBoxManage unregistervm <vm-name> --delete}

\section{How to add an extra hard disk to a VM?}
\subsection*{Ubuntu}
\begin{enumerate}
\item Turn sandbox off (see \ref{sandbox})
\item Shutdown VM (see \ref{stop})
\item Change Directory to VMs directory on host:\\
\cmd{cd /home/webis/VirtualBox VMs/<vm-name>}
\item Create a new disk with the following command:\\
\cmd{VBoxManage createvdi -filename extra-disk.vdi -size <size-in-meba-bytes>}

\item Add the new disk to the VM with the following command:\\
\cmd{VBoxManage storageattach <vm-name> --storagectl "SATA Controller" --port 1 --device 0 --type hdd --medium extra-disk.vdi}
\item Start VM (see \ref{start})
\item Install Software \emph{GParted} on VM and open it:\\
\cmd{sudo apt-get install gparted}\\
\cmd{sudo gparted}
\item Register extra disk with \emph{GParted}:
\begin{itemize}
\item Change to \emph{/dev/sbd1} in dropdown menu in the top left corner
\item Click \emph{Device: Create Partition Table}
\item Select \emph{/dev/sdb1} listed in the table and click \emph{New}:
\begin{itemize}
\item Typ: ext 4
\item Create as: Primary Partition
\end{itemize}
\item Click on the \emph{green checkmark} in the menu list (\emph{Apply Operations})
\end{itemize}
\item Create directory for extra disk on VM:\\
\cmd{sudo mkdir /media/extra-disk}
\item Open fstab file:\\
\cmd{sudo gedit /etc/fstab}\\
and add the following lines:\\
\cmd{\# extra disk}\\
\cmd{/dev/sdb1       /media/extra-disk       ext4    defaults             0      0}
\item Mount extra disk:\\
\cmd{sudo mount media/extra-disk}
\item Change permissions of extra-disk:\\
\cmd{sudo chown <user-name> /media/extra-disk}\\
\cmd{sudo chgrp sudo /media/extra-disk}
\end{enumerate}
    
\subsection*{Windows}
\begin{enumerate}
\item Turn sandbox off (see \ref{sandbox})
\item Change Directory to VMs directory on host:\\
\cmd{cd /home/webis/VirtualBox VMs/<vm-name>}
\item Execute the following command:\\
\cmd{VBoxManage createvdi -filename extra-disk.vdi -size <size-in-meba-bytes>}
\item Shutdown VM (see \ref{stop})
\item Execute the following command:\\
\cmd{VBoxManage storageattach <vm-name> --storagectl "SATA Controller" --port 1 --device 0 --type hdd --medium extra-disk.vdi}
\item Start VM (see \ref{start})
\item Open Control Panel $\rightarrow$ Disk Management:
\begin{itemize}
	\item Initialize with Master Boot Record (MBR)
	\item New Simple Volume
	\item Assign the following drive letter: X
	\item Volume label: extra-disk       
\end{itemize}
\end{enumerate}