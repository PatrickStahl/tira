\chapter{Submissions}

\section{How to add/remove a shared folder}

Add a shared folder with the following command:\\
\cmd{VBoxManage sharedfolder add <vm-name> --name <shared-folder-name> --hostpath <host-path> --readonly --automount}
\medskip\noindent

Remove a shared folderwith the following command:\\
\cmd{VBoxManage sharedfolder remove <vm-name> --name <shared-folder-name>}

\section{How to sandbox/unsandbox a VM?}\label{sandbox}
The sandbox script is placed on the host-pc in the directory tira-8/virtualbox/sandbox/.
One can sandbox a VM by executing the following command:\\
\cmd{./sandbox.sh <vm-name> on}\\
One can unsandbox a VM by extecuting the following command:\\
\cmd{./sandbox.sh <vm-name> off}\\
\medskip\noindent
In order to simplify the sandboxing procedure we prepare a wrapper script named sandbox-remote.sh. This script can be executed remotely from another pc than the host-pc. 
Moreover the sandbox-remote.sh need a submission file (see~\ref{submission-file}) as parameter. 
One can sandbox a VM remotely by executing the following command:\\
\cmd{./sandbox-remote.sh <submission-file> on}\\
One can unsandbox a VM remotely by executing the following command:\\
\cmd{./sandbox-remote.sh <submission-file> off}

\section{Unit-Test-Scripts}
A unit-test-script is supposed to be used to briefly check the submission of a participant. 
\lstinputlisting[style=BashOutputStyle]{example-unit-test.txt}

\section{Submission Files}\label{submission-file}
\lstinputlisting[style=BashOutputStyle]{example-submission.txt}

