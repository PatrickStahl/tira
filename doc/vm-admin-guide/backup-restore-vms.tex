\chapter{Backup/Restore VMs}

\section{How to backup a VM?}
The backup script is placed on the host-pc in the directory tira-8/virtualbox/backup/.
One can backup a VM by executing the following command:\\
\cmd{./backup-vm.sh <vm-name>}\\
The backup is an ova-file and will be save at\cmd{webis16} in the directory\\\cmd{/media/storage2/data-in-progress/workshop-PAN-13/pan13-virtual-machines-backup/}.\\
\medskip\noindent
In order to simplify the backup procedure we prepare a wrapper script named backup-vm-remote.sh. This script can be executed remotely from another pc than the host-pc. 
Moreover the backup-vm-remote.sh need a submission file (see~\ref{submission-file}) as parameter. 
One can backup a VM remotely by executing the following command:\\
\cmd{./backup-vm-remote.sh <submission-file> }


\section{How to copy a VM backup using checksums?}\label{copy-vm-backup}
Since the ova files are very large, please double-check the checksums after copying them to foreclose data loss. To do so, copy the ova file and place the checksum file next to it. 
The MD5 checksums for all VM backups are listed in the file~(see~\ref{checksum}):\\
\cmd{/media/storage2/data-in-progress/workshop-PAN-13/pan13-virtual-machines-backup/\\pan13-virtual-machines-backup-checksums.txt}.

\medskip\noindent
Then run the following command to test whether the file you copied is not corrupted:\\
\cmd{md5sum -c pan13-virtual-machines-backup-checksums.txt}

\medskip\noindent
The output of this command will be a list of all ova files you copied and their
state; for instance, it will look like this:\\
\cmd{pan13-aditya13-2013-07-09-12-21-00.ova: OK\\
pan13-agrawal13-2013-07-10-15-32-50.ova: OK\\
pan13-alshboul13-2013-07-10-15-12-22.ova: OK}

\section{How to restore a VMs backup}
The requierement for restoring a Vms backup is a prepared host machine~(see~\ref{prepare-host}). Furthermore, there are two points to know about restoring a VM's backup.

\medskip\noindent
First, because of the network settings, each VM is supposed to have its own host-only interface. Therefore, you need to create the host-only interface before restoring VM backups. Moreover, the assigned host-only interface corresponds to a VM's id, which is unfortunatly not readable in the ova filename. The simplest workaround is to create~20~(20~is the highest VM id) host-only interfaces by executing the following VirtualBox command 20~times:\\
\cmd{VBoxManage hostonlyif create}

\medskip\noindent
Second, in PAN 2013 we used 4 host machines for the VMs of the participants. As mentioned above, each VM used an individual host-only interface corresponding to its id. This means there were up to 4 VMs with the same host-only interface name but on different hosts. In order to avoid network conflicts it is important to only restore VMs with different ids on one host machine. In appendix \ref{mapping-hostonlyif-vms} the mapping of host-only interfaces to VMs is listed.

\medskip\noindent
Considering these two points you can easily restore a VMs backup on a prepared host (see \ref{prepare-host}):
\begin{enumerate}
\item Copy backup file to host machine~(see~\ref{copy-vm-backup})
\item Import ova file:\\
\cmd{VBoxManage import <backup.ova>}
\end{enumerate}


