\chapter{Host Machines}

Virtual machines are supposed to run one of these host-pc:
\begin{itemize}
\item \texttt{webis17.medien.uni-weimar.de} (141.54.159.7)
\begin{itemize}
\item RAM: 70 GB
\item Storage: 4.5 TB 
\item Processors: 16
\end{itemize}
\item \texttt{webis18.medien.uni-weimar.de} (141.54.159.8)
\begin{itemize}
\item RAM: 70 GB
\item Storage: 4.5 TB 
\item Processors: 16
\end{itemize}
\item \texttt{webis19.medien.uni-weimar.de} (141.54.159.9)
\begin{itemize}
\item RAM: 70 GB
\item Storage: 4.5 TB 
\item Processors: 16
\end{itemize}
\item \texttt{webis60.medien.uni-weimar.de} (141.54.159.10)
\begin{itemize}
\item RAM: 70 GB
\item Storage: 600 GB 
\item Processors: 16
\end{itemize}
\end{itemize}


\section{Connection}
Access the host-pc via ssh as webis user:\\
\cmd{ssh webis@<host-pc>}

\section{Filesystem}\label{filesystem}
All files necessary for VM administration are located on the host-pc in the directory \texttt{vm-conf}.\\

\dirtree{%
.1 home.
.2 webis.
.3 pan-13.
.4 pan13-training-data <shared folder>.
.5 <pan13-training-data..>.
.4 pan13-test-data <shared folder>.
.5 <pan13-test-data..>.
.3 tira-8.
.4 virtualbox.
.5 backup.
.6 backup-vm.sh.
.5 sandbox.
.6 sandbox.sh.
.6 $\sim$01-tira-ubuntu-12-04-desktop-64bit.lock.
.6 $\sim$10-tira-ubuntu-12-04-desktop-64bit.lock.
.6 $\dots$.
%.5 setup-files.
%.6 Oracle\_VM\_VirtualBox\_Extension\_Pack-4.1.12-77245.vbox-extpack.
%.6 virtualbox-4.1\_4.1.12-77245$\sim$Ubuntu$\sim$precise\_amd64.deb.
.5 \framebox{vm-conf}.
.6 configure-vm.py.
.6 configure-vm-ubuntu.sh.
.6 configure-vm-windows.sh.
.6 tira-ubuntu-12-04-desktop-64bit.ova.
.6 tira-windows-7-64bit.ova.
.6 vms.txt.
} 
\bigskip\noindent
The txt-file \texttt{vms.txt} provides a list of VM access information for VMs running on this host-pc.
Everytime you setup a VM with the configurations script \texttt{configure-vm.py} (see \ref{setup}) a new entry will be added in the \texttt{vms.txt} file.

\section{Prepare Host Machine}\label{prepare-host}
In order to prepare a host machine follow the instructions step by step.

\subsection{VirtualBox}
\begin{enumerate}
\item Enable virtualisation in bios of host pc
\item Install VirtualBox:
\begin{itemize}
	\item Download deb file:\\
	\cmd{download.virtualbox.org/virtualbox/4.2.10/virtualbox-4.2\_4.2.10-84104~Ubuntu~precise\_amd64.deb}
	\item Install deb file:\\
	\cmd{sudo dpkg --install <filename>.deb}
\end{itemize}

\item Install VirtualBox extension pack:
\begin{itemize}
	\item Download extpack:\\
	\cmd{download.virtualbox.org/virtualbox/4.2.10/Oracle\_VM\_VirtualBox\_Extension\_Pack-4.2.10-84104.vbox-extpack}
	\item Install extpack:\\
	\cmd{VBoxManage extpack install <filename>.vbox-extpack}
	\item Note: version number of virtual box and extension pack have to be identical
\end{itemize}

\item Link authentication library for rdp authentication:\\
\cmd{VBoxManage setproperty vrdeauthlibrary "VBoxAuthSimple"}
\end{enumerate}

\subsection{Additional software}
\begin{enumerate}
\item Install rdesktop and sshpass:\\
\cmd{sudo apt-get install rdesktop sshpass}
\end{enumerate}

\subsection{VM configuration data and scripts}
\begin{enumerate}
\item Copy ova-appliance-files (see~\ref{ova}) for import into virtual box to host pc (see~\ref{filesystem}):\\
\cmd{/home/webis/tira-8/virtualbox/vm-conf}.
\item Copy vm configuration scripts to host pc (see~\ref{filesystem}):\\
\cmd{/home/webis/tira-8/virtualbox/vm-conf}
\begin{itemize}
	\item configure-vm.py
	\item configure-vm-windows.sh
	\item configure-vm-ubuntu.sh
\end{itemize}
\end{enumerate}

\subsection{PAN training and test data}
\begin{enumerate}
\item Copy PAN training data into:\\
\cmd{/home/webis/pan-13/pan13-training-data}
\item Copy all PAN data (training and test data) into:\\
\cmd{/home/webis/pan-13/pan13-test-data}
\end{enumerate}

\subsection{Configure network}
\begin{enumerate}
\item Install dnsmasq:\\\cmd{sudo apt-get install dnsmasq}
\item Edit the last lines of \cmd{/etc/dnsmasq.conf} like this:
\lstinputlisting[style=BashOutputStyle,firstline=555,lastline=564]{dnsmasq.conf}
\item Add configuration file for VirtualBox (see \ref{dnsmasq-d-virtualbox}):\\
\cmd{/etc/dnsmasq.d/virtualbox}\\
Each line defines the corresponding IP for a host-only interface.
The IP address has the structure \cmd{10.18.4.100}, where
\begin{itemize}  \setlength{\parskip}{-6pt}
\item \cmd{18} is for the host \cmd{webis18}
\item \cmd{4} is for VM id 4, consequently for the host-only interface \cmd{vboxnet4}
\item \cmd{100} is the IP address of the VM
\end{itemize}
The address \cmd{10.18.4.1} would be the IP address of the interface.
\item Restart dnsmasq:\\\cmd{sudo service dnsmasq restart}

\item Configure iptables:\\\cmd{sudo bash iptables.sh}
\end{enumerate}
